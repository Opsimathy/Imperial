% --------------------------------------------------
% --- Preamble for the reference PDF ---
% --------------------------------------------------
\documentclass[11pt]{article}
\usepackage[a4paper,left=2cm,right=2cm,top=2cm,bottom=2cm]{geometry} 
\usepackage{amsmath}
\usepackage{amsthm}
\usepackage{amssymb}
\usepackage{graphicx}
\usepackage{mathtools}
\usepackage{hyperref}
\usepackage{bm}
\usepackage{enumitem}
\usepackage[dvipsnames]{xcolor}
\usepackage{ifthen}
\usepackage{helvet}

\renewcommand{\familydefault}{\sfdefault}
\renewcommand\thesubsection{\Roman{subsection} --}
\renewcommand\labelitemi{$\square$}
\parindent 0pt

%----------------------------------------------------------------------------------------
%	DEFINITIONS
%----------------------------------------------------------------------------------------

%%%%%%%%%%%%%%%%%%%%%%%%%%%%%%%%%%
% TO BE CHANGED FOR EACH STUDENT
\newcommand{\ModuleTitle}{Partial Differential Equations in Action}
\newcommand{\ModuleCode}{MATH50008}
\newcommand{\AuthorCID}{11111100110} 
\newcommand{\Date}{\today}
\newcommand{\Assignment}{Coursework 1}
%%%%%%%%%%%%%%%%%%%%%%%%%%%%%%%%%%

% headerline definition
\newcommand{\headerline}[6]{%
\par\medskip\noindent
\makebox[\textwidth][s]{\rlap{\large{{\bf #1}}}\hfill\llap{\large{{\bf #2}}}} \par
\vspace{0.5cm}
\makebox[\textwidth][s]{\rlap{\large{{\bf CID:} #3}}\hfill} \par
\makebox[\textwidth][s]{\rlap{\large{{\bf Date:} #4}}\hfill} \par
\vspace{0.5cm}
\makebox[\textwidth][c]{\large{{\bf #5}}}
\par\medskip}

% Useful mathematical notation definitions
\def\R{\mathbb{R}}
\def\0v{\mathbf{0}}
\def\uv{\mathbf{u}}
\def\vv{\mathbf{v}}
\def\wv{\mathbf{w}}
\def\rv{\mathbf{r}}
\def\pv{\mathbf{p}}
\def\av{\mathbf{a}}
\def\bv{\mathbf{b}}
\def\cv{\mathbf{c}}
\def\xv{\mathbf{x}}

\def\uvx{\hat{\bm{x}}}
\def\uvy{\hat{\bm{y}}}
\def\uvz{\hat{\bm{z}}}
\def\uvr{\hat{\bm{r}}}
\def\uvt{\hat{\bm{\theta}}}
\def\uvp{\hat{\bm{\phi}}}
\def\uvi{\hat{\mathbf{i}}}
\def\uvj{\hat{\mathbf{j}}}
\def\uvk{\hat{\mathbf{k}}}
\def\uvu{\hat{\mathbf{u}}}
\def\uvv{\hat{\mathbf{v}}}
\newcommand{\suv}[1]{\hat{\mathbf{e}}_{#1}}
\newcommand{\uuv}[1]{\mathbf{u}_{#1}}
\DeclareMathOperator{\sech}{sech}

\def\p{\partial}
\def\ux{\frac{\partial u}{\partial x}}
\def\uy{\frac{\partial u}{\partial y}}
\def\ut{\frac{\partial u}{\partial t}}
\def\uxx{\frac{\partial^2 u}{\partial x^2}}
\def\uyy{\frac{\partial^2 u}{\partial x^2}}
\def\uxy{\frac{\partial^2 u}{\partial x \partial y}}

\newcommand\underrel[3][]{\mathrel{\mathop{#3}\limits_{%
      \ifx c#1\relax\mathclap{#2}\else#2\fi}}}

%----------------------------------------------------------------------------------------
%	HYPERREFERENCES
%----------------------------------------------------------------------------------------
\PassOptionsToPackage{pdftex,hyperfootnotes=false,pdfpagelabels}{hyperref}
\usepackage{hyperref}
\pdfcompresslevel=9
\pdfadjustspacing=1

\hypersetup{
% Uncomment the line below to remove all links (to references, figures, tables, etc)
colorlinks=true, linktocpage=true, pdfstartpage=3, pdfstartview=FitV,
% Uncomment the line below if you want to have black links (e.g. for printing black and white)
%colorlinks=false, linktocpage=false, pdfborder={0 0 0}, pdfstartpage=3, pdfstartview=FitV, 
breaklinks=true, pdfpagemode=UseNone, pageanchor=true, pdfpagemode=UseOutlines,
plainpages=false, bookmarksnumbered, bookmarksopen=true, bookmarksopenlevel=1,
hypertexnames=true, pdfhighlight=/O, urlcolor=MidnightBlue, linkcolor=NavyBlue, citecolor=MidnightBlue,
}   
%----------------------------------------------------------------------------------------

% -----------------------------------
% --- Start of the main text ---
% -----------------------------------
\begin{document}

% -----------------
% --- Header ---
% -----------------
\begin{minipage}{0.3\textwidth}
\includegraphics[width=0.85\textwidth]{Imperial_logo.pdf}
\end{minipage}
\begin{minipage}{0.7\textwidth}
\centering
{\sc Department of Mathematics} \\
{\sc Imperial College London} \\
Academic Year 2023-2024
\end{minipage}
\vspace{1em}

% --------------------------------------------
% --- Change the header line here!
\headerline{\ModuleTitle}{\ModuleCode}{\AuthorCID}{\Date}{Student Answers to \Assignment}
% ---------------------------------------------

% --------------------------------------------
% Start of questions
% --------------------------------------------
\begin{enumerate}

% -----------------------------
\item If you want your equation to be numbered you can use the following environment 
\begin{equation}
u(x,t) = f(x,t)
\end{equation}
but if you prefer them not to be numbered just use the starred version of the environment
\begin{equation*}
u(x,t) = f(x,t)
\end{equation*}

If you need to go back to text inside the equation environment, use $\verb!\mbox{}!$ as follows
\begin{equation*}
\frac{du}{dt} = 0 \quad \mbox{on the characteristic defined by} \quad \frac{dx}{dt} = f(x)
\end{equation*}

For equations with multiple lines, the $\verb!align!$ environment is particularly useful.

% -----------------------------
\item \begin{enumerate}
\item 
\item 
\end{enumerate}
% -----------------------------
\item \begin{itemize}
\item 
\item 
\end{itemize}

\end{enumerate}
% --------------------------------------------
% End of questions
% --------------------------------------------

\end{document}
