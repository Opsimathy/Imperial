\documentclass{article}
\usepackage{graphicx}
\usepackage{enumitem}
\usepackage{physics}
\usepackage{amsmath}
\usepackage{amssymb}
\usepackage{braket}
\usepackage{xcolor}
\usepackage[a4paper, margin=1in]{geometry}
\title{Quantum Mechanics II, Coursework 1}
\author{Jiaru (Eric) Li (CID: 02216531)}
\date{12 February 2025}
\begin{document}
\maketitle
\section*{Question 1}
\begin{enumerate}[label=(\alph*)]
\item We are given that $\phi(x)=Ax\mathrm{e}^{-x^2/a^2}$, which must satisfy the normalisation condition $$\int^\infty_{-\infty}\abs{\phi(x)}^2\mathrm{d}x=1.$$

We have $\abs{\phi(x)}^2=A^2x^2\mathrm{e}^{-2x^2/a^2}$, and by setting $\alpha=2/a^2$, we get
\begin{align*}
\int^\infty_{-\infty}A^2x^2\mathrm{e}^{-2x^2/a^2}\mathrm{d}x&=A^2\int^\infty_{-\infty}x^2\mathrm{e}^{-\alpha x^2}\mathrm{d}x\\
&=-A^2\int^\infty_{-\infty}\frac{\mathrm{d}}{\mathrm{d}\alpha}\left(\mathrm{e}^{-\alpha x^2}\right)\mathrm{d}x\\
&=-A^2\frac{\mathrm{d}}{\mathrm{d}\alpha}\int^\infty_{-\infty}\mathrm{e}^{-\alpha x^2}\mathrm{d}x,
\end{align*}
where we recall the Gaussian integral $$\int^\infty_{-\infty}\mathrm{e}^{-\alpha x^2}\mathrm{d}x = \sqrt{\frac{\pi}{\alpha}}.$$

This gives $$\frac{\mathrm{d}}{\mathrm{d}\alpha}\int^\infty_{-\infty}\mathrm{e}^{-\alpha x^2}\mathrm{d}x=-\frac{\sqrt{\pi}}{2}\alpha^{-3/2}=-\frac{a^3\sqrt{\pi}}{2^{5/2}}.$$

Substituting into the normalisation condition gives
\begin{align*}
\left(-A^2\right)\left(-\frac{a^3\sqrt{\pi}}{2^{5/2}}\right)&=1\\
A^2&=\frac{2^{5/2}}{a^3\sqrt{\pi}},
\end{align*}

so we take $A$ to be $$\boxed{A=\frac{2^{5/4}}{a^{3/2}\pi^{1/4}}}.$$

\item To compute $\tilde{\phi}(p)$ in the momentum basis, we use the Fourier transform relation, namely
$$\tilde{\phi}(p)=\frac{1}{\sqrt{2\pi\hbar}}\int^\infty_{-\infty}\phi(x)\mathrm{e}^{-\mathrm{i}xp/\hbar}\mathrm{d}x.$$

Recall the standard definition of Fourier transform in the form of
$$\mathcal{F}\{f(x)\}(k)=\frac{1}{\sqrt{2\pi}}\int^\infty_{-\infty}f(x)\mathrm{e}^{-\mathrm{i}kx}\mathrm{d}x.$$

Therefore, by setting $k=p/\hbar$, we get
$$\tilde{\phi}(p)=\frac{1}{\sqrt{\hbar}}\mathcal{F}\{\phi(x)\}(k).$$

To compute the Fourier transform of $\phi(x)=Ax\mathrm{e}^{-x^2/a^2}$, we recall a property of Fourier transform
$$\mathcal{F}\{xf(x)\}(k)=\mathrm{i}\frac{\mathrm{d}}{\mathrm{d}k}\left(\mathcal{F}\{f(x)\}(k)\right),$$

and we already know the Fourier transform of a Gaussian function $f(x)=\mathrm{e}^{-\alpha x^2}$ as
$$\mathcal{F}\{f(x)\}(k)=\sqrt{\frac{1}{2\alpha}}\mathrm{e}^{-k^2/4\alpha}.$$

We then have
\begin{align*}
\mathcal{F}\{\phi(x)\}(k)&=\mathrm{i}A\frac{\mathrm{d}}{\mathrm{d}k}\left(\mathcal{F}\left\{\mathrm{e}^{-x^2/a^2}\right\}(k)\right)\\
&=\mathrm{i}A\frac{\mathrm{d}}{\mathrm{d}k}\left(\frac{a}{\sqrt{2}}\mathrm{e}^{-a^2k^2/4}\right)\\
&=\mathrm{i}A\frac{a}{\sqrt{2}}\mathrm{e}^{-a^2k^2/4}\left(-\frac{a^2k}{2}\right),
\end{align*}

so by plugging $A$ from part (a) in, we get
\begin{align*}
\mathcal{F}\{\phi(x)\}(k)&=-\mathrm{i}\frac{2^{5/4}}{a^{3/2}\pi^{1/4}}\frac{a}{\sqrt{2}}\frac{a^2}{2}\cdot k\mathrm{e}^{-a^2k^2/4}\\
&=-\mathrm{i}\frac{a^{3/2}}{\left(2\pi\right)^{1/4}}k\mathrm{e}^{-a^2k^2/4}\\
&=-\mathrm{i}\frac{a^{3/2}}{\hbar\left(2\pi\right)^{1/4}}p\mathrm{e}^{-a^2p^2/4\hbar^2},
\end{align*}

which gives the final answer
$$\boxed{\tilde{\phi}(p)=-\mathrm{i}\frac{\left(a/\hbar\right)^{3/2}}{\left(2\pi\right)^{1/4}}p\mathrm{e}^{-a^2p^2/4\hbar^2}}.$$

\item Again recall the Fourier transform relation between $\phi(x)$ and $\tilde{\phi}(p)$ given as
$$\tilde{\phi}(p)=\frac{1}{\sqrt{2\pi\hbar}}\int^\infty_{-\infty}\phi(x)\mathrm{e}^{-\mathrm{i}xp/\hbar}\mathrm{d}x.$$

Therefore, we have
$$\tilde{\phi}(-p)=\frac{1}{\sqrt{2\pi\hbar}}\int^\infty_{-\infty}\phi(x)\mathrm{e}^{\mathrm{i}xp/\hbar}\mathrm{d}x,$$

so if we substitute $x=-u$ into the integral and use the identity $\phi(x)=\phi(-x)$ given, we get
\begin{align*}
\tilde{\phi}(-p)&=\frac{1}{\sqrt{2\pi\hbar}}\int^{-\infty}_\infty\phi(-u)\mathrm{e}^{-\mathrm{i}up/\hbar}(-\mathrm{d}u)\\
&=\frac{1}{\sqrt{2\pi\hbar}}\int^\infty_{-\infty}\phi(u)\mathrm{e}^{-\mathrm{i}up/\hbar}\mathrm{d}u\\
&=\tilde{\phi}(p),
\end{align*}

which implies that $\boxed{\tilde{\phi}(p)\text{ is also even}}$.

\item We are now given the condition $\phi(x)=\left[\phi(-x)\right]^*$ where $a^*$ denotes the complex conjugate of $a$.

Recall again that
$$\tilde{\phi}(p)=\frac{1}{\sqrt{2\pi\hbar}}\int^\infty_{-\infty}\phi(x)\mathrm{e}^{-\mathrm{i}xp/\hbar}\mathrm{d}x.$$

The complex conjugate of $\tilde{\phi}(p)$ is given by
\begin{align*}
\left[\tilde{\phi}(p)\right]^*&=\left[\frac{1}{\sqrt{2\pi\hbar}}\int^\infty_{-\infty}\phi(x)\mathrm{e}^{-\mathrm{i}xp/\hbar}\mathrm{d}x\right]^*\\
&=\frac{1}{\sqrt{2\pi\hbar}}\int^\infty_{-\infty}\left[\phi(x)\mathrm{e}^{-\mathrm{i}xp/\hbar}\right]^*\mathrm{d}x\\
&=\frac{1}{\sqrt{2\pi\hbar}}\int^\infty_{-\infty}\left[\phi(x)\right]^*\mathrm{e}^{\mathrm{i}xp/\hbar}\mathrm{d}x,
\end{align*}

where we have used the identity $\left(e^z\right)^*=e^{z^*}$.

We can then substitute $x=-u$ into the integral, giving
\begin{align*}
\left[\tilde{\phi}(p)\right]^*&=\frac{1}{\sqrt{2\pi\hbar}}\int^{-\infty}_\infty\left[\phi(-u)\right]^*\mathrm{e}^{-\mathrm{i}up/\hbar}(-\mathrm{d}u)\\
&=\frac{1}{\sqrt{2\pi\hbar}}\int^\infty_{-\infty}\phi(u)\mathrm{e}^{-\mathrm{i}up/\hbar}\mathrm{d}u,
\end{align*}

which is precisely the definition of $\tilde{\phi}(p)$.

This implies that the complex conjugate of $\tilde{\phi}(p)$ equals to itself, i.e., $\boxed{\tilde{\phi}(p)\text{ is real}}$.
\end{enumerate}
\section*{Question 2}
\begin{enumerate}[label=(\alph*)]
\item The rotation operator $\hat{R}$ is given by $\hat{R}=\ket{3}\bra{1}+\ket{1}\bra{2}+\ket{2}\bra{3}$, so its Hermitian conjugate $\hat{R}^\dagger$ is $\hat{R}^\dagger=\ket{1}\bra{3}+\ket{2}\bra{1}+\ket{3}\bra{2}$.

By using the basic fact that $\braket{a|b}=1$ when $a=b$ and $0$ otherwise, we can then verify that
\begin{align*}
\hat{R}^\dagger\hat{R}&=(\ket{3}\bra{1}+\ket{1}\bra{2}+\ket{2}\bra{3})\cdot(\ket{1}\bra{3}+\ket{2}\bra{1}+\ket{3}\bra{2})\\
&=\ket{3}\braket{1|1}\bra{3}+\ket{3}\braket{1|2}\bra{1}+\ket{3}\braket{1|3}\bra{2}+\\
&\hspace{1.17em}\ket{1}\braket{2|1}\bra{3}+\ket{1}\braket{2|2}\bra{1}+\ket{1}\braket{2|3}\bra{2}+\\
&\hspace{1.17em}\ket{2}\braket{3|1}\bra{3}+\ket{2}\braket{3|2}\bra{1}+\ket{2}\braket{3|3}\bra{2}\\
&=\ket{3}\bra{3}+\ket{1}\bra{1}+\ket{2}\bra{2}=I.
\end{align*}
Similarly, the reflection operator $\hat{\sigma}=\ket{1}\bra{2}+\ket{2}\bra{1}+\ket{3}\bra{3}$ has Hermitian conjugate $\hat{\sigma}^\dagger=\ket{2}\bra{1}+\ket{1}\bra{2}+\ket{3}\bra{3}$ and satisfies
\begin{align*}
\hat{\sigma}^\dagger\hat{\sigma}&=(\ket{1}\bra{2}+\ket{2}\bra{1}+\ket{3}\bra{3})(\ket{2}\bra{1}+\ket{1}\bra{2}+\ket{3}\bra{3})\\
&=\ket{1}\braket{2|2}\bra{1}+\ket{1}\braket{2|1}\bra{2}+\ket{1}\braket{2|3}\bra{3}+\\
&\hspace{1.17em}\ket{2}\braket{1|2}\bra{1}+\ket{2}\braket{1|1}\bra{2}+\ket{2}\braket{1|3}\bra{3}+\\
&\hspace{1.17em}\ket{3}\braket{3|2}\bra{1}+\ket{3}\braket{3|1}\bra{2}+\ket{3}\braket{3|3}\bra{3}\\
&=\ket{1}\bra{1}+\ket{2}\bra{2}+\ket{3}\bra{3}=I.
\end{align*}
We have therefore shown that $\boxed{\hat{R}, \hat{\sigma}\text{ are unitary}}$.

To prove commutativity with the Hamiltonian $\hat{\mathcal{H}}$ given, recall that
$$\hat{\mathcal{H}}=-w[\ket{1}\bra{2}+\ket{2}\bra{3}+\ket{3}\bra{1}+\text{h.c.}].$$

Note that we have
\begin{align*}
\hat{R}\ket{1}&=(\ket{3}\bra{1}+\ket{1}\bra{2}+\ket{2}\bra{3})\ket{1}=\ket{3},\\
\hat{R}\ket{2}&=(\ket{3}\bra{1}+\ket{1}\bra{2}+\ket{2}\bra{3})\ket{2}=\ket{1},\\
\hat{R}\ket{3}&=(\ket{3}\bra{1}+\ket{1}\bra{2}+\ket{2}\bra{3})\ket{3}=\ket{2},
\end{align*}
so $\hat{R}$ is essentially a cyclic permutation that maps $\{1,2,3\}$ to $\{3,1,2\}$, leaving $\hat{\mathcal{H}}$ unchanged.

Intuitively, if we think of $\hat{\mathcal{H}}$ as describing a particle that can jump between vertices of an equilateral triangle, as given in the problem, then $\hat{R}$ is a $120^\circ$ rotation.

Similarly, we can also write
\begin{align*}
\hat{\sigma}\ket{1}&=(\ket{1}\bra{2}+\ket{2}\bra{1}+\ket{3}\bra{3})\ket{1}=\ket{2},\\
\hat{\sigma}\ket{2}&=(\ket{1}\bra{2}+\ket{2}\bra{1}+\ket{3}\bra{3})\ket{2}=\ket{1},\\
\hat{\sigma}\ket{3}&=(\ket{1}\bra{2}+\ket{2}\bra{1}+\ket{3}\bra{3})\ket{3}=\ket{3},
\end{align*}
so $\hat{\sigma}$ only swaps $1$ and $2$, which again preserves the symmetry of $\hat{\mathcal{H}}$.

Intuitively, $\hat{\sigma}$ is a reflection with the axis being the median from the vertex with label 3.

We have now proven that $\boxed{\hat{R}, \hat{\sigma}\text{ commutes with } \hat{\mathcal{H}}}$.

From the geometric picture given, and by looking at the structure of the Hamiltonian, we could find another two distinct unitary
symmetry operators, which are basically the other two reflections,
\begin{align*}
\hat{\sigma}_1&=\ket{2}\bra{3}+\ket{3}\bra{2}+\ket{1}\bra{1},\\
\hat{\sigma}_2&=\ket{1}\bra{3}+\ket{3}\bra{1}+\ket{2}\bra{2},
\end{align*}
in addition to the (trivial) identity operator.

\textit{(In fact, in the language of group theory, these symmetries form a dihedral group of order $6$, $D_6$. This group has $6$ elements: one identity; two $120^\circ$ rotations, clockwise and anticlockwise (corresponding to $\hat{R}$ and $\hat{R}^\dagger$); and three reflections. The fact that $D_6$ is isomorphic to the symmetric group of order 3, $S_3$, tells us about the symmetry of the Hamiltonian in an elegant way.)}

\item From part (a), we can write $\hat{R}$ and $\hat{\sigma}$ in matrix form with respect to the given basis $\{\ket{1},\ket{2},\ket{3}\}$ as permutation matrices,
$$\hat{R}=\begin{pmatrix}0&1&0\\0&0&1\\1&0&0\end{pmatrix},\quad\hat{\sigma}=\begin{pmatrix}0&1&0\\1&0&0\\0&0&1\end{pmatrix}.$$

To find the eigenvalues of $\hat{R}$, we compute
\begin{align*}
\det\left(\hat{R}-\lambda I\right)&=\begin{vmatrix}-\lambda&1&0\\0&-\lambda&1\\1&0&-\lambda\end{vmatrix}\\
&=-\lambda\begin{vmatrix}-\lambda&1\\0&-\lambda\end{vmatrix}-\begin{vmatrix}0&1\\1&-\lambda\end{vmatrix}\\
&=-\lambda^3+1,
\end{align*}
whose roots are the third roots of unity $\boxed{1,\omega,\omega^2}$ where $\omega=\mathrm{e}^{2\pi\mathrm{i}/3}$. These are the eigenvalues required.

The corresponding eigenvectors $v_i$ for $i=1,2,3$ must satisfy $\hat{R}v_i=\lambda v_i$, and so by letting $v_i=\begin{pmatrix}a\\b\\c\end{pmatrix}$, we get $\hat{R}v_i=\begin{pmatrix}c\\a\\b\end{pmatrix}$ and $\lambda v_i=\begin{pmatrix}\lambda a\\\lambda b\\\lambda c\end{pmatrix}$, and we have \begin{equation*}\left\{\begin{aligned}&c=\lambda a,\\&a=\lambda b,\\&b=\lambda c.\end{aligned}\right.\end{equation*}

For $\lambda=1$, one possible solution is $(a,b,c)=(1,1,1)$. For $\lambda=\omega$, one possible solution is $(a,b,c)=\left(1,\omega,\omega^2\right)$. For $\lambda=\omega^2$, one possible solution is $(a,b,c)=\left(1,\omega^2,\omega\right)$.

The eigenstates of $\hat{R}$ can therefore be expressed as
\begin{align*}
\ket{\phi_1}&=\ket{1}+\ket{2}+\ket{3},\\
\ket{\phi_2}&=\ket{1}+\omega\ket{2}+\omega^2\ket{3},\\
\ket{\phi_3}&=\ket{1}+\omega^2\ket{2}+\omega\ket{3},
\end{align*}

so an eigenbasis is given by $\boxed{\left\{\ket{\phi_1},\ket{\phi_2},\ket{\phi_3}\right\}}$.

Now we return to the Hamiltonian given,
$$\hat{\mathcal{H}}=-w[\ket{1}\bra{2}+\ket{2}\bra{3}+\ket{3}\bra{1}+\text{h.c.}].$$

We observe that this can alternatively be written as $\hat{\mathcal{H}}=-w\left(\hat{R}+\hat{R}^\dagger\right)=-w\left(\hat{R}+\hat{R}^2\right).$

Therefore, we have
\begin{align*}
\hat{\mathcal{H}}\ket{\phi_i}&=-w\left(\hat{R}+\hat{R}^2\right)\ket{\phi_i}\\
&=-w\left(\lambda+\lambda^2\right)\ket{\phi_i}.
\end{align*}

Since $\lambda+\lambda^2$ can take values of $1+1^2=2, \omega+\omega^2=-1$ and $\omega^2+\left(\omega^2\right)^2=-1$, the three eigenvalues of $\hat{\mathcal{H}}$ are given by $\boxed{-2w, w, w}$ under the same eigenbasis.

We then return our attention to $\hat{\sigma}=\begin{pmatrix}0&1&0\\1&0&0\\0&0&1\end{pmatrix}$. This matrix has eigenvalues $\mu$ such that
\begin{align*}
\det\left(\hat{\sigma}-\mu I\right)&=\begin{vmatrix}-\mu&1&0\\1&-\mu&0\\0&0&1-\mu\end{vmatrix}\\
&=-\mu\begin{vmatrix}-\mu&0\\0&1-\mu\end{vmatrix}-\begin{vmatrix}1&0\\0&1-\mu\end{vmatrix}\\
&=\mu^2(1-\mu)-(1-\mu)\\
&=-(1-\mu)^2(1+\mu),
\end{align*}
which has roots $\mu=1,1,-1$. These correspond to eigenvectors $u_i=\begin{pmatrix}a\\b\\c\end{pmatrix}$, so $\hat{\sigma}u_i=\begin{pmatrix}b\\a\\c\end{pmatrix}$ and $\mu u_i=\begin{pmatrix}\mu a\\\mu b\\\mu c\end{pmatrix}$, and we have \begin{equation*}\left\{\begin{aligned}&b=\mu a,\\&a=\mu b,\\&c=\mu c.\end{aligned}\right.\end{equation*}

For $\mu=1$, we have $a=b$ and $c$ is arbitrary. For $\mu=-1$, we need $a=-b$ and $c=0$. Therefore, the eigenstates of $\hat{\sigma}$ takes the form of $\ket{\psi_1}=\ket{1}-\ket{2}$ and $\ket{\psi_{2,3}}=a(\ket{1}+\ket{2})+b\ket{3}$, where $a$ and $b$ are constants.

We now compute
\begin{align*}
\hat{\mathcal{H}}\ket{\psi_{2,3}}&=-w[\ket{1}\bra{2}+\ket{2}\bra{3}+\ket{3}\bra{1}+\text{h.c.}][a(\ket{1}+\ket{2})+b\ket{3}]\\
&=-w(a\ket{1}\braket{2|1}+a\ket{1}\braket{2|2}+b\ket{1}\braket{2|3}+a\ket{2}\braket{3|1}+a\ket{2}\braket{3|2}+b\ket{2}\braket{3|3}+\\
&\hspace{2.48em}a\ket{3}\braket{1|1}+a\ket{3}\braket{1|2}+b\ket{3}\braket{1|3}+a\ket{2}\braket{1|1}+a\ket{2}\braket{1|2}+b\ket{2}\braket{1|3}+\\
&\hspace{2.48em}a\ket{3}\braket{2|1}+a\ket{3}\braket{2|2}+b\ket{3}\braket{2|3}+a\ket{1}\braket{3|1}+a\ket{1}\braket{3|2}+b\ket{1}\braket{3|3}\\
&=-w((a+b)\ket{1}+(a+b)\ket{2}+2a\ket{3}).
\end{align*}

Within the same eigenbasis, we have $\hat{\mathcal{H}}\ket{\psi_{2,3}}=\lambda\ket{\psi_{2,3}}$. By comparison, we get $\lambda a=-w(a+b)$ and $\lambda b=-2aw$. Recall that the three eigenvalues of $\hat{\mathcal{H}}$ are given by $-2w, w, w$. For $\lambda =-2w$, the solutions are $b=a$; for $\lambda=w$, the solutions are $b=-2a$.

The eigenstates of $\hat{\sigma}$ can therefore be expressed as
\begin{align*}
\ket{\psi_1}&=\ket{1}-\ket{2},\\
\ket{\psi_2}&=\ket{1}+\ket{2}+\ket{3},\\
\ket{\psi_2}&=\ket{1}+\ket{2}-2\ket{3},
\end{align*}

so an eigenbasis is given by $\boxed{\left\{\ket{\psi_1},\ket{\psi_2},\ket{\psi_3}\right\}}$.

\item We have already calculated that $$\hat{R}=\begin{pmatrix}0&1&0\\0&0&1\\1&0&0\end{pmatrix},\quad\hat{\sigma}=\begin{pmatrix}0&1&0\\1&0&0\\0&0&1\end{pmatrix},$$

so we could compute
$$\hat{R}\hat{\sigma}=\begin{pmatrix}1&0&0\\0&0&1\\0&1&0\end{pmatrix},\quad\hat{\sigma}\hat{R}=\begin{pmatrix}0&0&1\\0&1&0\\1&0&0\end{pmatrix},$$

and therefore
$$\left[\hat{R},\hat{\sigma}\right]=\hat{R}\hat{\sigma}-\hat{\sigma}\hat{R}=\begin{pmatrix}1&0&0\\0&0&1\\0&1&0\end{pmatrix}-\begin{pmatrix}0&0&1\\0&1&0\\1&0&0\end{pmatrix}=\boxed{\begin{pmatrix}1&0&-1\\0&-1&1\\-1&1&0\end{pmatrix}}.$$

Note that $\ket{\phi}=\ket{1}+\ket{2}+\ket{3}$ is a common eigenstate for both $\hat{R}$,$\hat{\mathcal{H}}$ and $\hat{\sigma}$,$\hat{\mathcal{H}}$. We have

$$\begin{pmatrix}1&0&-1\\0&-1&1\\-1&1&0\end{pmatrix}\begin{pmatrix}1\\1\\1\end{pmatrix}=\begin{pmatrix}0\\0\\0\end{pmatrix},$$

So we have $\boxed{\left[\hat{R},\hat{\sigma}\right]\ket{\phi}=0}$. This is consistent with Section 3.2 of the notes: while $\left[\hat{R},\hat{\sigma}\right]\neq0$, we do have $\left[\hat{R},\hat{\sigma}\right]\ket{\phi}=0$ for the non-degenerate eigenstate $\ket{\phi}$, verifying the relation between symmetry and degeneracy.
\end{enumerate}

\section*{Question 3}
\begin{enumerate}[label=(\alph*)]
\item We start with the 1d harmonic oscillator Hamiltonian $$\hat{\mathcal{H}}=\hbar\omega\left(\hat{a}^\dagger\hat{a}+\frac{1}{2}\right),$$

where the ladder operators $\hat{a}^\dagger$ and $\hat{a}$ satisfy $\left[\hat{a},\hat{a}^\dagger\right]=1$.

Therefore, we have $$\left[\hat{a},\hat{\mathcal{H}}\right]=\hbar\omega\left[\hat{a},\hat{a}^\dagger\hat{a}\right]=\hbar\omega\left(\left[\hat{a},\hat{a}^\dagger\right]\hat{a}+\hat{a}^\dagger\left[\hat{a},\hat{a}\right]\right)=\hbar\omega\hat{a}.$$

In the Heisenburg picture, the equation of motion for the ladder operator $\hat{a}_\mathrm{H}$ is given by
$$\mathrm{i}\hbar\frac{\mathrm{d}}{\mathrm{d}t}\hat{a}_\mathrm{H}(t)=\hat{\mathcal{U}}^\dagger\left[\hat{a},\hat{\mathcal{H}}\right]\hat{\mathcal{U}}=\hbar\omega\hat{\mathcal{U}}^\dagger\hat{a}\hat{\mathcal{U}}=\hbar\omega\hat{a}_\mathrm{H}(t),\quad\hat{a}_\mathrm{H}(0)=\hat{a},$$

which can be solved to give $\hat{a}_\mathrm{H}(t)=\mathrm{e}^{-\mathrm{i}\omega t}\hat{a}$. Thus, we also have $\hat{a}^\dagger_\mathrm{H}(t)=\mathrm{e}^{\mathrm{i}\omega t}\hat{a}^\dagger$.

Recall that we have $$\hat{a}\ket{\phi_0}=0,\quad\hat{a}\ket{\phi_1}=\ket{\phi_0},\quad\hat{a}^\dagger\ket{\phi_0}=\ket{\phi_1},\quad\hat{a}^\dagger\ket{\phi_1}=\sqrt{2}\ket{\phi_2},$$

where $\ket{\phi_0}$ and $\ket{\phi_1}$ are the ground and first excited states of the Hamiltonian.

Therefore, we have
$$\bra{\phi_0}\hat{a}\ket{\phi_0}=\bra{\phi_1}\hat{a}\ket{\phi_1}=\bra{\phi_1}\hat{a}\ket{\phi_0}=0,\quad\bra{\phi_0}\hat{a}\ket{\phi_1}=\braket{\phi_0|\phi_0}=1,$$
$$\bra{\phi_0}\hat{a}^\dagger\ket{\phi_0}=\bra{\phi_1}\hat{a}^\dagger\ket{\phi_1}=\bra{\phi_0}\hat{a}^\dagger\ket{\phi_1}=0,\quad\bra{\phi_1}\hat{a}^\dagger\ket{\phi_0}=\braket{\phi_1|\phi_1}=1.$$

Since the initial state is given as $\ket{\psi(0)}=\left(\ket{\phi_0}+\ket{\phi_1}\right)/\sqrt{2}$, the expectation value of $\hat{a}$ and $\hat{a}^\dagger$ at time $t$ can be found as
\begin{align*}
\braket{\hat{a}}(t)&=\bra{\psi(0)}\hat{a}_\mathrm{H}(t)\ket{\psi(0)}=\mathrm{e}^{-\mathrm{i}\omega t}\bra{\psi(0)}\hat{a}\ket{\psi(0)}=\frac{\mathrm{e}^{-\mathrm{i}\omega t}}{2}\bra{\left(\ket{\phi_0}+\ket{\phi_1}\right)}\hat{a}\ket{\left(\ket{\phi_0}+\ket{\phi_1}\right)}\\
&=\frac{\mathrm{e}^{-\mathrm{i}\omega t}}{2}\left(\bra{\phi_0}\hat{a}\ket{\phi_0}+\bra{\phi_1}\hat{a}\ket{\phi_1}+\bra{\phi_0}\hat{a}\ket{\phi_1}+\bra{\phi_1}\hat{a}\ket{\phi_0}\right)\\&=\frac{\mathrm{e}^{-\mathrm{i}\omega t}}{2}\left(0+0+1+0\right)=\frac{\mathrm{e}^{-\mathrm{i}\omega t}}{2},\\
\braket{\hat{a}^\dagger}(t)&=\bra{\psi(0)}\hat{a}^\dagger_\mathrm{H}(t)\ket{\psi(0)}=\mathrm{e}^{\mathrm{i}\omega t}\bra{\psi(0)}\hat{a}^\dagger\ket{\psi(0)}=\frac{\mathrm{e}^{\mathrm{i}\omega t}}{2}\bra{\left(\ket{\phi_0}+\ket{\phi_1}\right)}\hat{a}^\dagger\ket{\left(\ket{\phi_0}+\ket{\phi_1}\right)}\\
&=\frac{\mathrm{e}^{\mathrm{i}\omega t}}{2}\left(\bra{\phi_0}\hat{a}^\dagger\ket{\phi_0}+\bra{\phi_1}\hat{a}^\dagger\ket{\phi_1}+\bra{\phi_0}\hat{a}^\dagger\ket{\phi_1}+\bra{\phi_1}\hat{a}^\dagger\ket{\phi_0}\right)\\&=\frac{\mathrm{e}^{\mathrm{i}\omega t}}{2}\left(0+0+0+1\right)=\frac{\mathrm{e}^{\mathrm{i}\omega t}}{2}.
\end{align*}

Recall that the position operator $\hat{x}$ and the momentum operator $\hat{p}$ can be represented as
$$\hat{x}=\sqrt{\frac{\hbar}{2m\omega}}(\hat{a}+\hat{a}^\dagger),\quad\hat{p}=-\mathrm{i}\sqrt{\frac{m\hbar\omega}{2}}(\hat{a}-\hat{a}^\dagger).$$

By linearity of expectation, we could therefore determine the expectation of $\hat{x}$ as
$$\braket{\hat{x}}(t)=\sqrt{\frac{\hbar}{2m\omega}}\left(\braket{\hat{a}}(t)+\braket{\hat{a}^\dagger}(t)\right)=\sqrt{\frac{\hbar}{2m\omega}}\left(\frac{\mathrm{e}^{-\mathrm{i}\omega t}}{2}+\frac{\mathrm{e}^{\mathrm{i}\omega t}}{2}\right)=\boxed{\sqrt{\frac{\hbar}{2m\omega}}\cos{\omega t}},$$

where we have used the identities $\mathrm{e}^{\mathrm{i}\omega t}=\cos\omega t+\mathrm{i}\sin\omega t$ and $\mathrm{e}^{-\mathrm{i}\omega t}=\cos\omega t-\mathrm{i}\sin\omega t$.

\item Similarly, we have $$\braket{\hat{p}}(t)=-\mathrm{i}\sqrt{\frac{m\hbar\omega}{2}}\left(\braket{\hat{a}}(t)-\braket{\hat{a}^\dagger}(t)\right)=-\mathrm{i}\sqrt{\frac{m\hbar\omega}{2}}\left(\frac{\mathrm{e}^{-\mathrm{i}\omega t}}{2}-\frac{\mathrm{e}^{\mathrm{i}\omega t}}{2}\right)=-\sqrt{\frac{m\hbar\omega}{2}}\sin{\omega t}.$$

To determine the variance, we also need to calculate the expectation of $\hat{x}^2$ and $\hat{p}^2$.

We have
\begin{align*}
\hat{x}^2&=\frac{\hbar}{2m\omega}\left(\hat{a}+\hat{a}^\dagger\right)^2=\frac{\hbar}{2m\omega}\left(\hat{a}^2+\hat{a}\hat{a}^\dagger+\hat{a}^\dagger\hat{a}+\hat{a}^{\dagger 2}\right),\\
\hat{p}^2&=-\frac{m\hbar\omega}{2}\left(\hat{a}-\hat{a}^\dagger
\right)^2=-\frac{m\hbar\omega}{2}\left(\hat{a}^2-\hat{a}\hat{a}^\dagger-\hat{a}^\dagger\hat{a}+\hat{a}^{\dagger 2}\right).
\end{align*}

Note that, by definition, we have
$$\hat{a}^2\ket{\phi_0}=0,\quad\hat{a}^2\ket{\phi_1}=\hat{a}\ket{\phi_0}=0,\quad\hat{a}^{\dagger 2}\ket{\phi_0}=\hat{a}^\dagger\ket{\phi_1}=\ket{\phi_2},\quad\hat{a}^{\dagger 2}\ket{\phi_1}=\sqrt{2}\hat{a}^\dagger\ket{\phi_2}=\sqrt{6}\ket{\phi_3},$$

so we must have $\braket{\hat{a}^2}(t)=0$ since both states result in $0$ and $\braket{\hat{a}^{\dagger 2}}(t)=0$ for any time $t$ since both states are outside of (orthogonal to) the original states.

For the number operator $\hat{a}^\dagger\hat{a}$, we have
$$\hat{a}^\dagger\hat{a}\ket{\phi_0}=0,\quad\hat{a}^\dagger\hat{a}\ket{\phi_1}=\ket{\phi_1},\quad\left(\hat{a}^\dagger\hat{a}\right)_\mathrm{H}=\hat{a}^\dagger_\mathrm{H}\hat{a}_\mathrm{H}(t)=\mathrm{e}^{\mathrm{i}\omega t}\hat{a}^\dagger\mathrm{e}^{-\mathrm{i}\omega t}\hat{a}=\hat{a}^\dagger\hat{a},$$

so its expectation is
\begin{align*}
\braket{\hat{a}^\dagger\hat{a}}(t)&=\bra{\psi(0)}\left(\hat{a}^\dagger\hat{a}\right)_\mathrm{H}\ket{\psi(0)}=\bra{\psi(0)}\hat{a}^\dagger\hat{a}\ket{\psi(0)}=\frac{1}{2}\bra{\left(\ket{\phi_0}+\ket{\phi_1}\right)}\hat{a}^\dagger\hat{a}\ket{\left(\ket{\phi_0}+\ket{\phi_1}\right)}\\
&=\frac{1}{2}\left(\bra{\phi_0}\hat{a}^\dagger\hat{a}\ket{\phi_0}+\bra{\phi_1}\hat{a}^\dagger\hat{a}\ket{\phi_1}+\bra{\phi_0}\hat{a}^\dagger\hat{a}\ket{\phi_1}+\bra{\phi_1}\hat{a}^\dagger\hat{a}\ket{\phi_0}\right)\\&=\frac{1}{2}\left(0+1+0+0\right)=\frac{1}{2}.
\end{align*}

By the identity $\left[\hat{a},\hat{a}^\dagger\right]=\hat{a}\hat{a}^\dagger-\hat{a}^\dagger\hat{a}=1$, we have
$$\braket{\hat{a}\hat{a}^\dagger}(t)=1+\braket{\hat{a}^\dagger\hat{a}}(t)=1+\frac{1}{2}=\frac{3}{2}.$$

Finally, we are able to compute 
\begin{align*}
\braket{\hat{x}^2}(t)&=\frac{\hbar}{2m\omega}\left(\braket{\hat{a}^2}(t)+\braket{\hat{a}\hat{a}^\dagger}(t)+\braket{\hat{a}^\dagger\hat{a}}(t)+\braket{\hat{a}^{\dagger 2}}(t)\right)=\frac{\hbar}{2m\omega}\left(0+\frac{3}{2}+\frac{1}{2}+0\right)=\frac{\hbar}{m\omega},\\
\braket{\hat{p}^2}(t)&=-\frac{m\hbar\omega}{2}\left(\braket{\hat{a}^2}(t)-\braket{\hat{a}\hat{a}^\dagger}(t)-\braket{\hat{a}^\dagger\hat{a}}(t)+\braket{\hat{a}^{\dagger 2}}(t)\right)=-\frac{m\hbar\omega}{2}\left(0-\frac{3}{2}-\frac{1}{2}+0\right)=m\hbar\omega.
\end{align*}

The variance the position and momentum at time $t$ are, respectively,
\begin{align*}
\sigma_x^2(t)&=\braket{\hat{x}^2}(t)-\left(\braket{\hat{x}}(t)\right)^2=\frac{\hbar}{m\omega}-\left(\sqrt{\frac{\hbar}{2m\omega}}\cos{\omega t}\right)^2=\boxed{\frac{\hbar}{m\omega}\left(1-\frac{1}{2}\cos^2{\omega t}\right)},\\
\sigma_p^2(t)&=\braket{\hat{p}^2}(t)-\left(\braket{\hat{p}}(t)\right)^2=m\hbar\omega-\left(-\sqrt{\frac{m\hbar\omega}{2}}\sin{\omega t}\right)^2=\boxed{m\hbar\omega\left(1-\frac{1}{2}\sin^2{\omega t}\right)}.
\end{align*}

The product between these variances is
\begin{align*}
\sigma_x^2(t)\sigma_p^2(t)&=\frac{\hbar}{m\omega}\left(1-\frac{1}{2}\cos^2{\omega t}\right)\cdot m\hbar\omega\left(1-\frac{1}{2}\sin^2{\omega t}\right)\\&=\frac{\hbar^2}{4}\left(2-\cos^2{\omega t}\right)\left(2-\sin^2{\omega t}\right),
\end{align*}

and because $\cos^2{\omega t},\sin^2{\omega t}\leq1$ for $t\in\mathbb{R}$, we have
\begin{align*}
\sigma_x^2(t)\sigma_p^2(t)&\geq\frac{\hbar^2}{4}\left(2-1\right)\left(2-1\right)=\frac{\hbar^2}{4},
\end{align*}

which is indeed $\boxed{\text{consistent with the Heisenberg uncertainty principle}}$.

In fact, for this case, we can calculate the lower bound of the variance product $\sigma_x^2(t)\sigma_p^2(t)$ by using $\cos^2 x+\sin^2 x=1$ where $x=\omega t$, so that
\begin{align*}
\left(2-\cos^2x\right)\left(2-\sin^2x\right)&=(2-\cos^2x)(1+\cos^2x)\\
&=2+\cos^2x-\cos^4x=\frac{9}{4}-\left(\cos^2x-\frac{1}{2}\right)^2,
\end{align*}

which is minimised when $\cos^2x=0$ or $\cos^2x=1$ since $0\leq\cos^2x\leq1$, with minimum $9/4-1/4=2$.

This corresponds to the cases where $x=k\pi$ or $x=k\pi+\pi/2$ where $k\in\mathbb{Z}$, i.e., $$t=\frac{k\pi}{\omega}\text{ or }\frac{k\pi+\pi/2}{\omega}.$$

At these times, the variance product is minimised at $\boxed{\left(\sigma_x^2(t)\sigma_p^2(t)\right)_{\min}=\hbar^2/2}$.
\end{enumerate}
\end{document}